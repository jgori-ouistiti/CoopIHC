\documentclass[12pt,a4paper]{article}
\usepackage[utf8]{inputenc}
\usepackage[english]{babel}
\usepackage{amsmath}
\usepackage{amsfonts}
\usepackage{amssymb}
\usepackage{graphicx}
\usepackage{lmodern}
\usepackage[left=3cm,right=2cm,top=2cm,bottom=2cm]{geometry}

\usepackage{tikz}
	\usetikzlibrary{calc}
	\usetikzlibrary{shapes}
	\usetikzlibrary{arrows}
	\usetikzlibrary{fit}

\usepackage{here}

\title{Interaction-Agents}
\author{Julien Gori}
\begin{document}
\maketitle

Code used below:
\begin{itemize}
\item \textbf{bold} is implemented and working
\item plain text is work in progress/close to working
\item \textit{italic} is nothing done yet
\end{itemize}
	

\bigskip
\section{Reasons to use interaction-agents}

\begin{enumerate}
\item An unified API for designing interactive systems: components designed by different people will be compatible and thus interoperable. This promises better cooperation between research teams (alleviates the need to design every single component, which in turn encourages the development of complex components), and more transparent evaluations (by having common and identical baselines with which to compare). Open AI's gym has boosted RL research in similar ways.
\item A modular library: many components can be reused as subcomponents for other more complex components
\end{enumerate}


\section{Use cases for interaction-agents}

\begin{enumerate}

\item Developing user models
	\begin{itemize}
	\item Likelihood-based models 
		\begin{itemize}
		\item theoretical model $\longrightarrow$ \textbf{1D Pointing example}
		\end{itemize}
	\item Likelihood free models
		\begin{itemize}
		\item  \textit{supervised/unsupervised learning} 
		\item RL $\longrightarrow$ \textbf{1D pointing example}, Xiuli's eye-gaze selection model
		\end{itemize}
	\item 
		\end{itemize}
\item Evaluating assistants 
	\begin{itemize}
	\item on synthetic user models $\longrightarrow$ Comparing BIGPointer to some other assistant for two different user models (user with and \textbf{without} cost for observation due to cursor jumps)
	\item \textit{with real users}
		\begin{itemize}
		\item \textit{Create a task that is not Python-based. Possibly Node.js application, that can exchange information with a python app, that would be wrapped in an agent class to be bundled.}
		\end{itemize}
	\end{itemize}
\item Developing smart assistants (which handle user models)
	\begin{itemize}
	\item BIGPointer with cost for observation, trained via RL
	\item \textit{AI-Assisted design} (Sebastiaan)
		\begin{itemize}
		\item \textit{interface with more computationally efficient languages e.g. Julia via PyJulia}
		\end{itemize}
	\end{itemize}
	
\item Synthetic psychology (Andrew)
	\begin{itemize}
	\item Assistant which handles user models + \textit{Bayesian Optimal Experimental Design}
	\end{itemize}
	
\item Simulating adaptation 
	\begin{itemize}
	\item \textit{Two agent RL}
	\item mixed RL (assistant)/ adaptive-model-based (user)
		\begin{itemize}
		\item 2D Continuous Pointing
		\end{itemize}
	\end{itemize}
\end{enumerate}

\section{Discussion with Antti}
\begin{itemize}
\item bringing inference into the discussion is a reason to use bundles, see paper The frontier of simulation-based inference
\item bundles allow a style of design of interactions that is similar to sim-to-real, see Sim-to-Real Transfer in Deep Reinforcement Learning for Robotics: a Survey
\item new android env gym see AndroidEnv: A Reinforcement LearningPlatform for Android. Would be useful to train user models, but not assistants.
\item user models with degree of freedom. Assistant has a tradeoff between exploiting the model, or trying to refine the model. See Aurélien's work on machine teaching. Question of experiment design.
\item Discussion with Markku: create web apps with web sockets. Tornado can be used on Python side.
\item Discussion with Aini: Ideas for Christoph Johns:1) "check if he can implement the ten simple rules" 2) write a task using web socket for Aini's work 3) rewrite some components e.g. observation engine
\end{itemize}

\section{Arguments}

\begin{itemize}
\item Having strong user models that the assistant can leverage is likely required (see the paper Occam's razor is insufficient to infer the preferences of irrational agents).
\item ``computational models instantiate \textit{algorithmic hypotheses} about how behavior is generated'' (wilson2019)
\end{itemize}



\end{document}